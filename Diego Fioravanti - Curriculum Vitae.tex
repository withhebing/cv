\documentclass[italian,a4paper]{europasscv}
\usepackage[italian]{babel}
\ecvname{Diego Fioravanti}
\ecvaddress{Viale Europa 111, 38057 Pergine Valsugana, TN, Italia}
\ecvtelephone{+39 3280067080}
\ecvemail{fioravanti.diego@gmail.com}
\ecvhomepage{https://www.linkedin.com/in/diegofioravanti https://github.com/diravan}
\ecvdateofbirth{29/07/1989}
\ecvnationality{Italiana}
\ecvgender{Maschio}
\ecvpicture[width=1in]{picture.jpg}
\ecvpictureleft

\hypersetup{
    pdfauthor={Diego Fioravanti},
    pdftitle={Diego Fioravanti - Curriculum Vitae},
    pdfsubject={Curriculum Vitae},
    pdfkeywords={Curriculum Vitae, CV, Diego Fioravanti, Diego, Fioravanti}
}

\begin{document}
	\begin{europasscv}
	\ecvpersonalinfo
	
 	\ecvsection{ISTRUZIONE E FORMAZIONE}
  
		\ecvtitlelevel{2013 -- 2015}{Laurea magistrale in matematica}{Laurea magistrale}
			\ecvitem{}{Università di Trento, dipartimento di matematica, Trento (Italia)}
        		\ecvitem{}{Voto: 107/110}
            	\ecvitem{}{Descrizione: i corsi intrapresi in questa laurea si concentrano su argomenti di matematica pura e computazione simbolica, particolarmente nelle aree di geometria algebrica e algebra commutativa.}
			\ecvitem{}{Tesi: ``Cohomology and spectral sequences in algebraic geometry''}
			\ecvitem{}{Relatore: Prof. Edoardo Ballico.}

  		\ecvtitle{01/09/2013 -- 30/06/2014}{Erasmus}
			\ecvitem{}{Utrecht University, Utrecth (Paesi Bassi)}	
			\ecvitem{}{Descrizione: Erasmus di un anno, i corsi frequentati riguardavano geometria algebrica, algebra commutativa e programmazione parallela.}
		
		\ecvtitlelevel{2008 -- 2012}{Laurea triennale in matematica}{Laurea Triennale}
			\ecvitem{}{Università di Trento, dipartimento di matematica, Trento (Italia)}
        		\ecvitem{}{Voto: 98/110}
            \ecvitem{}{Descrizione: i corsi intrapresi in questa laurea si concentrano su argomenti di matematica pura, in particolare algebra ed aspetti fondazionali, e programmazione, in particolare algoritmi, programmazione funzionale e programmazione scientifica.}
         	\ecvitem{}{Tesi: ``Il teorema di Artin-Wedderburn''}
			\ecvitem{}{Relatore: Prof. Willem A. de Graaf.}
			
    \ecvsection{ ESPERIENZA PROFESSIONALE }

		\ecvtitle{2009 -- Presente}{Insegnante privato di matematica ed informatica}
					\ecvitem{}{Trento (Italia)}
			\ecvitem{}{Insegnante privato di matematica ed informatica per scuole medie e superiori ed università a Trento. L'insegnamento a livello di scuole dell'obbligo si concentra sul curriculum standard previsto dal ministero. L'insegnamento a livello universitario si concentra su corsi del primo e secondo anno, corsi come: analisi I e II, algebra lineare e introduzione alla programmazione. Ho esperienza come insegnante sia con un singolo studente sia con gruppi da due a dieci studenti contemporaneamente.}

		\ecvtitle{08/2015 -- 06/2016}{Insegnante privato di matematica}
					\ecvitem{}{Oslo, Norvegia}
					\ecvitem{}{Insegnante privato di matematica ed informatica per scuole medie e superiori ed università a Oslo, Norvegia. L'insegnamento a livello di scuole dell'obbligo era basato attorno al curriculum IB (International Baccalaureate). L'insegnamento a livello universitario si concentra su corsi del primo e secondo anno, corsi come: analisi I e II e algebra lineare.}   
		\ecvtitle{09/2011 -- 12/2011}{Tirocinio}
			\ecvitem{}{Istituto d'istruzione Marie Curie, Pergine Valsugana (Italia)}
			\ecvitem{}{Esperienza di tirocinio in una scuola superiore affiancato da un gruppo di tre insegnanti. Durante il tirocinio mi è stato richiesto di preparare lezioni, esporle agli studenti e di preparare e correggere temi su tali lezioni.}  	
			
    \ecvsection{Lingue}
			\ecvmothertongue{Italiano}
			\ecvlanguageheader
				\ecvlanguage{Inglese}{C2}{C2}{C2}{C2}{C2}
				\ecvlanguagecertificate{TOEFL (Punteggio 100)}
				\ecvlastlanguage{Norvegese (Bokmål)}{B2}{B2}{B1}{B1}{B1}
				\ecvlastlanguage{Tedesco}{A1}{A1}{A1}{A1}{A1}
	        \ecvlanguagefooter  	
	        
	\ecvsection{Competenze tecniche}
	
 		\ecvblueitem{Linguaggi di programmazione e tecnologie}{
			\begin{ecvitemize}
	  			\item Base: Java, HTML, CSS, Javascript, QT5
	  			\item Intermedio: Haskell, Django, Matlab/Octave, (Neo)Vim, GIMP, metodologia TDD, SQL
	  			\item Avanzato: C/C++, Python, \LaTeX, Linux
	  		\end{ecvitemize}
  		}
  
		\ecvblueitem{Progetti}{
		  \begin{ecvitemize}
		    \item Computazione parallela di $\pi$ (2014). Progetto finale per il corso ``parallel computing'', il programma calcola $1054039$ cifre corrette di $\pi$ usando un algoritmo parallelo. C, Python.
		    \item Crivello di Eratostene (2013). Progetto per il corso ``parallel computing'', il programma eseguiva il crivello di Eratostene in parallelo. C, Python.
		    \item L'algoritmo di Artin - Wedderburn (2012). Parte della tesi triennale, il programma  computed the Artin - Wedderburn decomposition of semisimple algebras. Python.
  \end{ecvitemize}
  		}
  		

    \ecvsection{Competenze personali}		
		\ecvblueitem{Competenze comunicative}{
			\begin{ecvitemize}
				\item Lavoro in gruppo: I lavorato in gruppi di varie dimensioni durante il mio percorso di studio. Nel corso del mio stage ho lavorato a stretto contatto con un gruppo di tre insegnanti. 
                \item Intermediazione: Durante le mie esperienze come insegnante ho insegnato a gruppi composti da studenti con diversi livelli di capacità e conoscenze sempre riuscendo ad accomodare le necessità del singolo all'interno delle necessità complessive del gruppo.
                \item Competenze culturali: Sono abituato a lavorare e studiare in un ambiente europeo ed internazionale grazie alle mie esperienze all'estero, Erasmus in Olanda ed insegnante privato in Norvegia.
          \end{ecvitemize}
       }
  	
        \ecvblueitem{Competenze organizzative e gestionali}{
            \begin{ecvitemize}
                \item Durante entrambe le mie lauree ho organizzato, prodotto e presentato una serie di lezioni informali su come scrivere una tesi in \LaTeX.
                \item Durante le mie esperienze come insegnate mi sono trovato ad organizzare e gestire orari e/o piani di studio per singoli studenti e gruppi di studenti.
            \end{ecvitemize}
        }  	
        
        \ecvblueitem{Altre competenze}{            
            \begin{ecvitemize}
                \item Fotografo amatoriale.
            \end{ecvitemize}
        }
            
        \ecvblueitem{Patente}{B}

        \ecvsection{Privacy}
  
        \ecvblueitem{}{Autorizzo il trattamento dei dati personali contenuti nel mio curriculum vitae in base art. 13 del D. Lgs. 196/2003.}
     
  	\end{europasscv}
\end{document}